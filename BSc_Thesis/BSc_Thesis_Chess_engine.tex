\documentclass[10pt, a4paper]{report}

% ============================================================
% PAKIETY KODOWANIA I JĘZYKOWE
% ============================================================
\usepackage[utf8]{inputenc} % Kodowanie znaków
\usepackage[T1]{fontenc}    % Kodowanie fontów (ważne dla polskich znaków)
\usepackage[polish]{babel}  % Polskie dzielenie wyrazów i napisy (Rozdział, Spis treści)

% ============================================================
% 1. USTAWIENIA STRONY (Marginesy)
% ============================================================
% Lewy 3.5cm, reszta 2.5cm
\usepackage[left=3.5cm, right=2.5cm, top=2.5cm, bottom=2.5cm]{geometry}

% ============================================================
% 2. CZCIONKA I INTERLINIA
% ============================================================
\usepackage{fontspec}      % Pakiet do obsługi fontów systemowych
\setmainfont{Verdana}      % Ustawiamy Verdanę jako główny font
\newfontfamily\Verdana{Verdana} % Na wszelki wypadek

% Interlinia 1.5 (wymagane 1.15 lub 1.5)
\usepackage{setspace}
\onehalfspacing 

\usepackage{graphicx}

% ============================================================
% 3. AKAPITY I FORMATOWANIE TEKSTU
% ============================================================
% Wcięcie akapitowe 1.25 cm
\setlength{\parindent}{1.25cm} 

% Domyślnie LaTeX NIE robi wcięcia w pierwszym akapicie po tytule 
% (co jest zgodne z Twoim wymaganiem).
% Pakiet babel-polish może czasem wymuszać wcięcie wszędzie, 
% ale standardowa klasa 'report' powinna to nadpisać. 
% Gdyby jednak wcięcia pojawiły się pod tytułami, ten kod poniżej by to naprawiał:
% \usepackage{titlesec} 
% (ale na razie zostawiamy standard, bo jest poprawny typograficznie)

% Justowanie tekstu jest włączone domyślnie w LaTeX.

% ============================================================
% 4. LICZBY I JEDNOSTKI
% ============================================================
\usepackage{siunitx}
% Konfiguracja: polski przecinek jako separator dziesiętny
\sisetup{output-decimal-marker = {,}} 

% ============================================================
% POCZĄTEK DOKUMENTU
% ============================================================
\begin{document}

% --- Strona Tytułowa (uproszczona) ---
\newgeometry{left=2.5cm, right=2.5cm, top=2.5cm, bottom=2.5cm}
\begin{titlepage}
    \centering

    \includegraphics[height=5.5cm]{graphics/agh_znk_wbr_rgb_150ppi.jpg} 
    
    \vspace{0.5cm}
    
    % {\fontsize{24}{28}\selectfont \bfseries \sffamily AGH} \par
    
    % \vspace{0.5cm}

    {\bfseries \sffamily \fontsize{14}{16}\selectfont AKADEMIA GÓRNICZO-HUTNICZA IM. STANISŁAWA STASZICA W KRAKOWIE} \par
    \vspace{0.2cm}
    {\bfseries \sffamily WYDZIAŁ GEOLOGII, GEOFIZYKI I OCHRONY ŚRODOWISKA} \par
    \vspace{0.2cm}
    {\sffamily KATEDRA GEOINFORMATYKI I INFORMATYKI STOSOWANEJ} \par

    \vspace{3cm} % Duży odstęp przed tytułem

    % 2. TYTUŁ PRACY
    {\Large \sffamily Praca dyplomowa} \par
    
    \vspace{1.5cm}
    
    % Tytuł polski (kursywa)
    {\Large \itshape \bfseries
    Projekt silnika szachowego z wykorzystaniem metod \\
    sztucznej inteligencji
    } \par
    
    \vspace{0.5cm}
    
    % Tytuł angielski (kursywa)
    {\large \itshape
    Chess engine design using artificial intelligence methods
    } \par

    \vspace{3cm} % Odstęp przed danymi autora

    % 3. DANE AUTORA I OPIEKUNA
    % Używamy tabeli, żeby ładnie wyrównać dwukropek
    \begin{flushleft}
        % \hspace{4cm} % Przesunięcie tabeli w prawo (dostosuj, jeśli chcesz inaczej)
        \begin{tabular}{ll}
            Autor: & \textbf{Bartłomiej Kózka} \\ 
            Kierunek studiów: & \textit{Inżynieria i Analiza Danych} \\
            Opiekun pracy: & \textit{Prof. dr hab. inż. Norbert Skoczylas} \\
        \end{tabular}
    \end{flushleft}

    \vfill % Wypycha datę na sam dół strony

    % 4. MIEJSCE I ROK
    Kraków, 2025

\end{titlepage}
\restoregeometry
% LaTeX automatycznie nie numeruje strony tytułowej (liczy ją jako 1, ale nie drukuje numeru)

% --- Spis treści ---
\tableofcontents

% --- Wstawienie Spisu Ilustracji ---
\newpage % Wymusza nową stronę
\addcontentsline{toc}{chapter}{Spis ilustracji} % Ręczne dodanie wpisu do spisu treści
\listoffigures % Wygenerowanie listy obrazków

% --- Właściwa treść pracy ---

\chapter{Wstęp}
To jest pierwszy akapit pod tytułem rozdziału. Zgodnie z Twoimi wymaganiami, **nie powinien on mieć wcięcia**. LaTeX robi to automatycznie w standardzie typografii anglosaskiej i naukowej. Tekst jest wyjustowany (wyrównany do lewej i prawej).

A to jest drugi akapit. Tutaj już widzisz wcięcie wynoszące dokładnie 1,25 cm. LaTeX sam dba o podział wyrazów w języku polskim (dzięki pakietowi \texttt{babel}).

\section{Przykłady liczb i jednostek}
Tutaj pierwszy akapit pod sekcją – również bez wcięcia. Poniżej przykłady użycia pakietu \texttt{siunitx} do liczb:

Akapit z wcięciem. Masa próbki wynosiła \num{12.5} (zauważ, że w kodzie wpisuję kropkę, a w PDF drukuje się przecinek – to zasługa ustawień). 
Długość to \SI{55}{\milli\meter}. Pakiet ten automatycznie dba o "twardą spację" między liczbą a jednostką.

Możesz też pisać ręcznie używając tyldy jako twardej spacji, np.: 
Długość wynosi 10~km. (Znak tyldy \texttt{\textasciitilde} w LaTeX to właśnie twarda spacja – Ctrl+Shift+Spacja z Worda).

\chapter{Cel pracy}
Tekst w nowym rozdziale. Znowu brak wcięcia na początku. Margines lewy jest szerszy (3,5 cm) na oprawę pracy, a pozostałe mają 2,5 cm.

\chapter{Zasady gry w szachy}

\chapter{Projekt silnika szachowego}

\chapter{Użyte technologie}

\chapter{Implementacja silnika}

\chapter{Analiza rezultatów i wydajności poszczególnych wersji silnika}

\chapter{Wnioski}

\newpage
% \addcontentsline... (BibTeX zazwyczaj sam dodaje wpis, a jak nie, to wtedy dodasz)
\bibliographystyle{plain}
\bibliography{nazwa_twojego_pliku_bib}

\end{document}
