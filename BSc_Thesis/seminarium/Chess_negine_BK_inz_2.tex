% This example uses the 'minted' package, so you need to run the compiler with 
% the '-shell-escape' option, i.e.
% 	pdflatex -shell-escape example.tex
% Additionally, you must have the 'pygmentize' program installed, which is part of 
% the 'Pygments' package (https://pygments.org/)

\documentclass[aspectratio=1610, polish]{beamer} 
% Jeśli chcesz otrzymać prezentację w języku polskim, to, powyżej, zamień „english” na „polish”
\usepackage{babel}
\makeatletter
\@ifclasswith{beamer}{polish}{
	\usepackage{polski}
}
\makeatother
\usepackage[utf8]{inputenc}
\usepackage{listings} % We want to put listings
\usepackage{minted}   % We want to put listings
\usepackage{graphicx} % We want to put images
\usepackage{pifont}


\renewcommand{\familydefault}{\rmdefault}


\mode<beamer>{ 	% In the 'beamer' mode
	\definecolor{links}{HTML}{2A1B81}
	\hypersetup{
		pdfpagemode=FullScreen,                 % Enable Full screen mode
		colorlinks,
		linkcolor=,
		urlcolor=links
	}
	\usetheme[parttitle=rightfooter]{AGH}       % Show part title in right footer
	%\usetheme[nosidebar]{AGH}                  % Do not show sidebar on non-title slides
	%\usetheme[nosidebar, margins=1em]{AGH}     % Do not show sidebar on non-title slides and set both margins (left / right) to 1em
	%\usetheme[dark]{AGH}                       % Use dark background
	%\usetheme[dark, parttitle=leftfooter]{AGH} % Use dark background and show part title in left footer
}
\mode<handout>{	% In the 'handout' mode
	\hypersetup{pdfpagemode=None}		
	\usepackage{pgfpages}
	\pgfpagesuselayout{4 on 1}[a4paper,border shrink=5mm,landscape]	% Show 4 slides on 1 page
	\pgfpageslogicalpageoptions{1}{border code=\pgfusepath{stroke}}
	\pgfpageslogicalpageoptions{2}{border code=\pgfusepath{stroke}}
	\pgfpageslogicalpageoptions{3}{border code=\pgfusepath{stroke}}
	\pgfpageslogicalpageoptions{4}{border code=\pgfusepath{stroke}}
  	\usetheme{boxes}
  	\addheadbox{structure}{\quad\insertpart\hfill\insertsection\hfill\insertsubsection\qquad}          % Content of header
	\addfootbox{structure}{\quad\insertshortauthor\hfill\insertframenumber\hfill\insertsubtitle\qquad} % Content of footer
	  }

\AtBeginPart{ % At begin part: display its name
	\frame{\partpage}
}

\author[Bartłomiej Kózka]
{
		Bartłomiej Kózka \newline
        Promotor: Skoczylas Norbert, prof. dr hab. inż.
}
\date{}

\title[Silnik szachowy]
{
    Projekt silnika szachowego z wykorzystaniem metod sztucznej inteligencji -  \textcolor{red}{2}
}


%%%%%%%%%%% Configuration of the listings package %%%%%%%%%%%%%%%%%%%%%%%%%%
% Source: https://en.wikibooks.org/wiki/LaTeX/Source_Code_Listings#Using_the_listings_package
%%%%%%%%%%%%%%%%%%%%%%%%%%%%%%%%%%%%%%%%%%%%%%%%%%%%%%%%%%%%%%%%%%%%%%%%%%%%
\lstset{ %
  backgroundcolor=\color{white},   % choose the background color
  basicstyle=\footnotesize,        % the size of the fonts that are used for the code
  breakatwhitespace=false,         % sets if automatic breaks should only happen at whitespace
  breaklines=true,                 % sets automatic line breaking
  captionpos=b,                    % sets the caption-position to bottom
  commentstyle=\color{green},      % comment style
  deletekeywords={...},            % if you want to delete keywords from the given language
  escapeinside={\%*}{*)},          % if you want to add LaTeX within your code
  extendedchars=true,              % lets you use non-ASCII characters; for 8-bits encodings only, does not work with UTF-8
  frame=single,	                   % adds a frame around the code
  keepspaces=true,                 % keeps spaces in text, useful for keeping indentation of code (possibly needs columns=flexible)
  keywordstyle=\color{blue},       % keyword style
  morekeywords={*,...},            % if you want to add more keywords to the set
  numbers=left,                    % where to put the line-numbers; possible values are (none, left, right)
  numbersep=5pt,                   % how far the line-numbers are from the code
  numberstyle=\tiny\color{gray},   % the style that is used for the line-numbers
  rulecolor=\color{black},         % if not set, the frame-color may be changed on line-breaks within not-black text (e.g. comments (green here))
  showspaces=false,                % show spaces everywhere adding particular underscores; it overrides 'showstringspaces'
  showstringspaces=false,          % underline spaces within strings only
  showtabs=false,                  % show tabs within strings adding particular underscores
  stepnumber=2,                    % the step between two line-numbers. If it's 1, each line will be numbered
  stringstyle=\color{cyan},        % string literal style
  tabsize=2,	                   % sets default tabsize to 2 spaces
  title=\lstname,                  % show the filename of files included with \lstinputlisting; also try caption instead of title
                                   % needed if you want to use UTF-8 Polish chars
  literate={ą}{{\k{a}}}1
           {Ą}{{\k{A}}}1
           {ę}{{\k{e}}}1
           {Ę}{{\k{E}}}1
           {ó}{{\'o}}1
           {Ó}{{\'O}}1
           {ś}{{\'s}}1
           {Ś}{{\'S}}1
           {ł}{{\l{}}}1
           {Ł}{{\L{}}}1
           {ż}{{\.z}}1
           {Ż}{{\.Z}}1
           {ź}{{\'z}}1
           {Ź}{{\'Z}}1
           {ć}{{\'c}}1
           {Ć}{{\'C}}1
           {ń}{{\'n}}1
           {Ń}{{\'N}}1
}
%%%%%%%%%%% Configuration of the minted package %%%%%%%%%%%%%%%%%%%%%%%%%%
\setminted[C++]{frame=single,linenos}

%%%%%%%%%%%%%%%%%
\begin{document}
\maketitle

\begin{frame}{Plan prezentacji}
	\tableofcontents
\end{frame}

%%%%%%%%%%%%%%%%

\section{Cel pracy}

\begin{frame}{Cel pracy}
    \begin{itemize}
        \item \textbf{Głównym celem pracy} jest zaprojektowanie i implementacja silnika szachowego zdolnego do samodzielnej gry na poziomie \textbf{1000 ELO}, z wykorzystaniem klasycznych technik przeszukiwania drzewa gry (drzewa decyzyjnego).
        \newline
        \item \textbf{Celem dodatkowym} jest porównanie skuteczności różnych funkcji oceny oraz metod optymalizacji procesu wyszukiwania ruchów. \textcolor{red}{\ding{55}}  

    \end{itemize}
\end{frame}


%%%%%%%%%%%%%%%%

\section{Metodyka i Wyniki}
%----------------
\begin{frame}{Metodyka}
    % - wspomioeinen o wytworzeniu wlansych "bibliotek" np. do bit operatoion funkcje, generowanie magics numbers
    \begin{enumerate}
    \item \textbf{Analiza teoretyczna i przegląd rozwiązań} (zapoznanie z istniejącymi silnikami) \textcolor{green}{\ding{51}} 
    
    \item \textbf{Projekt architektury silnika} \textcolor{green}{\ding{51}} % wybór reprezentacji planszy, podział na moduły (plansza, generacja ruchów, wyszukiwanie, ocena).\textcolor{green}{\ding{51}} 
    
    \item \textbf{Konfiguracja środowiska i pipeline CI/CD} \textcolor{green}{\ding{51}} 

    \item \textbf{Implementacja i optymalizacja algorytmu przeszukiwania} \textcolor{orange}{\ding{117}}
    
    \item \textbf{Funkcja oceny pozycji} (opracowanie heurystyki oceny) \textcolor{green}{\ding{51}} 
        
    \item \textbf{Weryfikacja i ocena działania} (testy poprawności, porównanie wyników z innymi silnikami)  \textcolor{orange}{\ding{117}}
    
    \item \textbf{Wnioski i dalsze kierunki rozwoju}  \textcolor{orange}{\ding{117}}
    \end{enumerate}   

\end{frame}

%----------------

\begin{frame}{Metodyka – Reprezentacja planszy oraz Generacja ruchów \textcolor{green}{\ding{51}} }
    % Wspomnienie o wykorzystaniu operacji bitowych dla zwiększenia szybkości działania,
    % brak złożonej abstrakcji i polimorfizmu w czasie wykonania → low-level programming (C++).
    Wybrana metoda reprezentacji planszy:
    \newline
    \begin{itemize}
        \item \textbf{Piece-centric (bitboards)} – plansza traktowana jako skończony zbiór 64 pól,
        najczęściej reprezentowany przy użyciu 64-bitowych liczb całkowitych (każde pole odpowiada jednemu bitowi).
    \end{itemize}

    \vspace{1.5em}
    Wybrana metoda Generacji ruchów:
    \newline
    \begin{itemize}
        \item \textbf{Iteracyjna (generatorowa)} – leniwe” generowanie ruchów i ich bieżąca ocena.
    \end{itemize}
\end{frame}

%----------------

\begin{frame}{Metodyka – Reprezentacja planszy oraz Generacja ruchów \textcolor{green}{\ding{51}} }
    Ocena poprawności i szybkości generacji ruchów – \textbf{testy perft}
        \vspace{1.0em}

    \includegraphics[scale=0.3]{perft.png}

\end{frame}

\begin{frame}{Wyniki – Reprezentacja planszy oraz Generacja ruchów \textcolor{green}{\ding{51}} }

    Przykład wykonania perft testów
    \begin{itemize}
        \item \textbf{Głębokość}: 6
        \item \textbf{Liczba odwiedzonych node'ów}: 120 mln.
        \item \textbf{Czas wykonania}: ~ 15 sekund
    \end{itemize}
    \begin{center}
    \includegraphics[scale=0.39]{perft_res.png}
    \end{center}

\end{frame}

%----------------

\begin{frame}[fragile]{Metodyka – Algorytm przeszukiwania oraz Funkcja Oceny \textcolor{orange}{\ding{117}}}
    % Ocena z punktu widzenia gracza A = – ocena z punktu widzenia gracza B.
    % Algorytm przeszukuje drzewo gry w głąb, stosując negację wyniku dla przeciwnika.
    Wybrana metoda przeszukiwania ruchów:
    \vspace{1.0em}
    \begin{itemize}
        \item \textbf{NegaMax framework}, Alpha–beta pruning
    \end{itemize}
    \vspace{1.0em}
    Wybrana funkcja oceny: (Hand-Crafted Evaluation)
    \vspace{1.0em}
    \begin{lstlisting}[
  basicstyle=\ttfamily\scriptsize,
  numbers=none,
  aboveskip=2pt, belowskip=2pt,
  columns=fullflexible,
  xleftmargin=0pt, framexleftmargin=0pt, framesep=2pt
]
    f(p) = 200(K - K')
         + 9(Q - Q')
         + 5(R - R')
         + 3(B - B' + N - N')
         + 1(P - P')
         - 0.5(D - D' + S - S' + I - I')
         + 0.1(M - M') + ...

    KQRBNP = liczba króli, hetmanów, wież, gońców, skoczków i pionków
    D, S, I = zdwojone, zablokowane i izolowane pionki
    M = mobility (liczba legalnych ruchów)
    \end{lstlisting}
\end{frame}

%----------------

\begin{frame}[fragile]{Wyniki - CI/CD \textcolor{green}{\ding{51}}}
    \textbf{CI}: builds and run tests
    \begin{center}
    \includegraphics[scale=0.35]{ci.png}
    \end{center}
\end{frame}


\section{Pozostałe}
\begin{frame}{Pozostałe \textcolor{orange}{\ding{117}}}
    \begin{itemize}
        \item \textbf{Finalizacja implementacji alg. przeszukiwania}
        \item \textbf{Ocena wyników}
        \begin{itemize}
            \item Analiza statycznej funkcji oceny - porównanie oceny danej pozycji z oceną innego silnika (stockfish)
            \item Testy gry przeciw innym silnikom - oszacowanie finalnej siły silnika
        \end{itemize}
    \end{itemize}
\end{frame}

%%%%%%%%%%%%%%%%

\section{Narzędzia}
%----------------
\begin{frame}{Narzędzia}
    \begin{itemize}
        \item język programowania: C++
        \item kompilacja: GCC, CMake
        \item środowisko: Visual Studio Code
        \item kontrola wersji: Git 
        \item CI/CD: GitHub Actions
        \item testy jednostkowe: GoogleTest framework
        \item testy funkcjonalne: stockfish (perft), python-chess framework
        \noindent\rule{\textwidth}{0.5pt}
        \item interfejs użytkownika: UCI (Universal Chess Interface) – 
            możliwość integracji z lokalnymi lub internetowymi GUI (np. Arena, CuteChess, Lichess) \textcolor{green}{\ding{51}}
    \end{itemize}
\end{frame}

%%%%%%%%%%%%%%%%

\section{Źródła}
%----------------
\begin{frame}{Źródła}
    \begin{itemize}
        \item Dokumnetacja wymienionych narzędzi (m.in. https://cppreference.com)
        \newline
        \item Chess Programming Wiki – internetowa encyklopedia poświęcona programowaniu silników szachowych. (https://www.chessprogramming.org)
    \end{itemize}
\end{frame}
%----------------
\begin{frame}{}
    \centering
    \Huge Dziękuję za uwagę.
    \vspace{1cm}
\end{frame}


\end{document}
