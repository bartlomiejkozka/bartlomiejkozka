\documentclass[9pt]{extarticle}

\usepackage[utf8]{inputenc}  % UTF-8 support
\usepackage[T1]{fontenc}     % Better font encoding for Computer Modern
\usepackage{lipsum}          % For sample text (you can remove this)
\usepackage[
    top=    2cm,   
    bottom= 1cm, 
    left=   2cm,    
    right=  2cm, 
]{geometry}
\usepackage{hyperref}
\usepackage{enumitem}

\pagestyle{empty}

\begin{document}

% personal information
\begin{center}
{\LARGE{\textbf{Bartłomiej Kózka}}} \\
\vspace{0.2cm}
Krakow, Poland - bartlomiej.kozka.bis@gmail.com - +48 607 551 824 - 
    \href{https://www.linkedin.com/in/bart%C5%82omiejkozka44/}{linkedin link} -
    \href{https://github.com/bartlomiejkozka}{github link}
\end{center}

% ------------------------------------------------------------------------------

\subsection*{PROFFESIONAL EXPERIENCE}
\vspace{-1.5em}
\rule{\textwidth}{0.1pt}
\vspace{-0.5em}

\noindent\textbf{HID Global - Firmware Engineer Intern}
\begin{itemize}[itemsep=-3pt, topsep=3pt]
    \item \textbf{Signo Reader} project - secure access control software.
    \item Actively contributed to the full Scrum development cycle. 
          \textbf{Implemented} inter-process communication using Protocol Buffers.
          \textbf{Designed and Implemented} continuous integration (CI) pipelines.
    \item Technologies: C/C++, Python, UNIX, unit testing (CUnit), functional testing (pytest), docker, jenkins, low-level design
\end{itemize}
%-------------
\noindent\textbf{AGH Solar Boat Team - Software Engineer}
\begin{itemize}[itemsep=-3pt, topsep=3pt]
    \item Contributed to the development of an autonomous boat using ROS2, C++, and Python, implementing a scalable three-layer architecture with modules for object clustering, sensor fusion, tracking, and obstacle avoidance.
\end{itemize}

% ------------------------------------------------------------------------------

\subsection*{EDUCATION}
\vspace{-1.5em}
\rule{\textwidth}{0.1pt}
\vspace{-0.5em}

\noindent
\makebox[\textwidth]{%
\textbf{The AGH University of Krakow} \hfill October 2022 - Present \\
}
B.Sc. \textbf{Data Science}, 3-year GPA \textbf{4.75 / 5.0}

% ------------------------------------------------------------------------------

\subsection*{PERSONAL PROJECTS}
\vspace{-1.5em}
\rule{\textwidth}{0.1pt}
\vspace{-0.5em}

\noindent
\makebox[\textwidth]{%
\textbf{Barkoz Tempo - Chess Engine} \hfill 2025 \\
}
\textit{Link}
\href{https://github.com/bartlomiejkozka/Barkoz-Tempo}{https://github.com/bartlomiejkozka/Barkoz-Tempo}
\begin{itemize}[itemsep=-3pt, topsep=3pt]
    \item Currently developing a \textbf{C++} chess engine based on the \textbf{Minimax algorithm} with \textbf{Alpha-Beta pruning} and \textbf{bitboard representation} for efficient state encoding.
\end{itemize}

\noindent
\makebox[\textwidth]{%
\textbf{Gear shift decision maker - Gear shift decision maker based on AI Fuzzy Logic algorithm} \hfill 2025 \\
}
\textit{Link}
\href{https://github.com/bartlomiejkozka/Gear-shift-FuzzyLogic}{https://github.com/bartlomiejkozka/Gear-shift-FuzzyLogic}
\begin{itemize}[itemsep=-3pt, topsep=3pt]
    \item Designed and implemented an \textbf{AI-based fuzzy logic} system in \textbf{C++} to simulate automatic gear shifting in vehicles based on real-time inputs such as velocity, RPM, and throttle level, building a rule-based decision-making engine with defuzzification techniques to model realistic driving behavior and visualize gear transitions under various conditions.
\end{itemize}

\noindent
\makebox[\textwidth]{%
\textbf{Tiny Compiler - Pseudo-code compiler} \hfill 2024 \\
}
\textit{Link}
\href{https://github.com/bartlomiejkozka/teeny-tiny-compiler}{https://github.com/bartlomiejkozka/teeny-tiny-compiler}
\begin{itemize}[itemsep=-3pt, topsep=3pt]
    \item Developed a lightweight pseudo-code compiler application using \textbf{Python} and \textbf{FastAPI}
          framework, structured as a \textbf{REST API} with integrated database functionality.
          Enables users to learn coding by writing, compiling, and executing pseudo-code, with the
          ability to save and manage scripts within their personal accounts
\end{itemize}

%---------------------------------------------
\vspace{0.2cm}
\noindent\textbf{\underline{Data Science}}
\vspace{0.1cm}

\noindent
\makebox[\textwidth]{\textbf{Analysis of Bicycle Traffic on the Fremont Bridge in Seattle} \hfill 2025 \\} 
    \textit{Link} \href{https://github.com/bartlomiejkozka/Fremont-bike-analysis}{https://github.com/bartlomiejkozka/Fremont-bike-analysis} 
    \begin{itemize}[itemsep=-3pt, topsep=3pt]
    \item Analyzed bicycle traffic data from Seattle’s Fremont Bridge using \textbf{Python (pandas, matplotlib, seaborn)} \\
        to identify temporal patterns and effectively communicate insights through visualizations.
\end{itemize}

\noindent
\makebox[\textwidth]{\textbf{Brenna Meteorological Data - EDA} \hfill 2025 \\}
    \textit{Link} \href{https://github.com/bartlomiejkozka/Brenna-meteorology-2020}{https://github.com/bartlomiejkozka/Brenna-meteorology-2020}
\begin{itemize}[itemsep=-3pt, topsep=3pt]
    \item Performed \textbf{exploratory data analysis} on 2020 weather data from Brenna using pandas, matplotlib, seaborn \\
        to uncover seasonal patterns, correlations, and anomalies in temperature, humidity, and precipitation.
\end{itemize}
%---------------------------------------------

% ------------------------------------------------------------------------------

% \subsection*{HACKATHONS}
% \vspace{-1.5em}
% \rule{\textwidth}{0.1pt}
% \vspace{-0.5em}

% \noindent
% \makebox[\textwidth]{%
% \textbf{\underline{HackYeah Krakow}} \hfill 2024 \\
% }
% \begin{itemize}[itemsep=-3pt, topsep=3pt]
%     \item Co-developed a wellness application - Sleep Diary, focused on analyzing sleep patterns to improve sleep quality.
% \end{itemize}

% ------------------------------------------------------------------------------

\subsection*{SKILLS}
\vspace{-1.5em}
\rule{\textwidth}{0.1pt}
Python, R, C, C++, Bash, Algorithms and Data Structures, 
Data analysis \& visualization, Feature engineering, Exploratory Data Analysis,
Machine learning, Predictive analytics (regression, classification), Unsupervised learning (clustering)
Multi-Threaded Programming, UNIX Systems, TCP, UDP, Sockets and Networking Protocols, POSIX Libraries, 
gtest, CUnit, Docker, Jenkins, CMake, OOP and Design Patterns, Git, SonarQube  
Agile, Scrum, Jira, Ability to work on and debug unfamiliar code

% ------------------------------------------------------------------------------

\subsection*{LANGUAGES}
\vspace{-1.5em}
\rule{\textwidth}{0.1pt}
English (B2), Polish (Native)

% ------------------------------------------------------------------------------
\newpage
% ------------------------------------------------------------------------------

\subsection*{AGREEMENTS}
\vspace{-1.5em}
\rule{\textwidth}{0.1pt}
I agree to the processing of personal data provided in this document for realising the recruitment process pursuant to the
Personal Data Protection Act of 10 May 2018 (Journal of Laws 2018, item 1000) and in agreement with Regulation (EU)
2016/679 of the European Parliament and of the Council of 27 April 2016 on the protection of natural persons with regard
to the processing of personal data and on the free movement of such data, and repealing Directive 95/46/EC (General
Data Protection Regulation).

% ------------------------------------------------------------------------------


\end{document}
